\documentclass{article}

\usepackage{amsmath}
\usepackage{amssymb}
\usepackage{enumitem}
\usepackage[margin=1in]{geometry}

% Commands
\newcommand{\term}[1]{\textbf{\emph{#1}}}
\newcommand{\mm}[1]{\(#1\)}

% Theorems
\newtheorem{definition}{Definition}[section]

% List

\newlist{axioms}{enumerate}{1}
\setlist[axioms]{label=(\arabic*), leftmargin=4em}


\title{Least Upper Bound - Supremum}
\author{Joseph Song}
\date{}

\begin{document}
\maketitle

\section{Axiom of Completeness}
\textbf{Axiom of Completeness} - Every nonempty set of real numbers that is bounded above has a least upper bound.
\begin{definition}[Upper bound]
    Set $A \subseteq R$ is bounded above if there exists a real number $b$ such that it is greater than all the values in A. ($\exists\ b \in \mathbb{R}: a \leq b, \forall\ a\in A$)
\end{definition}

\begin{definition}[Least Upper Bound - Supremum]
    The least upper bound is an element s in A, such that:
    \begin{enumerate}
        \item s is an upper bound
        \item For any other upper bound, b, $s \leq b$
    \end{enumerate}
    This means that out of all the bounds, s is the smallest. The supremum is denoted as $s = \operatorname{sup}(A)$.
\end{definition}

\begin{example}
    For a set A and a number $c \in \mathbb{R}$, define
    \begin{center}
        $c + A = \{c + a : a \in A\}$
    \end{center}
    and show $\operatorname{sup}(c+A) = c+\operatorname{sup}A$.
\end{example}
\begin{proof}
    
\end{proof}

\end{document}