\documentclass{article}

\usepackage{amsmath}
\usepackage{amssymb}
\usepackage{enumitem}
\usepackage[margin=1in]{geometry}

% Commands
\newcommand{\term}[1]{\textbf{\emph{#1}}}
\newcommand{\mm}[1]{\(#1\)}

% Theorems
\newtheorem{definition}{Definition}[section]

% List

\newlist{axioms}{enumerate}{1}
\setlist[axioms]{label=(\arabic*), leftmargin=4em}


\title{Least Upper Bound - Supremum}
\author{Joseph Song}
\date{}

\begin{document}
\maketitle

\section{Axiom of Completeness}
\textbf{Axiom of Completeness} - Every nonempty set of real numbers that is bounded above has a least upper bound.
\begin{definition}[Upper bound]
    Set $A \subseteq R$ is bounded above if there exists a real number $b$ such that it is greater than all the values in A. ($\exists\ b \in \mathbb{R}: a \leq b, \forall\ a\in A$)
\end{definition}

\begin{definition}[Least Upper Bound - Supremum]
    The least upper bound is an element s in A, such that:
    \begin{enumerate}
        \item s is an upper bound
        \item For any other upper bound, b, $s \leq b$
    \end{enumerate}
    This means that out of all the bounds, s is the smallest. The supremum is denoted as $s = \operatorname{sup}(A)$.
\end{definition}

\begin{example}
    For a set A and a number $c \in \mathbb{R}$, define
    \begin{center}
        $c + A = \{c + a : a \in A\}$
    \end{center}
    and show $\operatorname{sup}(c+A) = c+\operatorname{sup}A$.
\end{example}
\begin{proof}

    Let $s$ be the least upper bound of $\operatorname{sup}(A)$. By definition, this means that $a \leq s, \forall\ a \in A$. Addting $c \implies c +a \leq c + s$. Notice that the $c + a$ for all a is just the set $c + A$, so $c + s \implies c+ supA$ is an upper bound. For (2), let b be another arbitrary upper bound for c + A, meaning $b \geq c + a$. $b \geq c + a \implies b - c \geq a$, so $b - c$ is an upper bound for set A. Since we assumed $s$ is the least upper bound, $s \leq b - c \implies c + s \leq b$, but b is an upper bound for c + A, therefore $c + s$ is the least upper bound.
    
\end{proof}

\begin{example}
    Given two nonempty, bounded above sets , A and B, define $A + B = \{a + b : a \in A\ and\ b\in B\}$. Prove $\operatorname{sup}(A+B) = \operatorname{sup}A + \operatorname{sup}B$.
\end{example}
\begin{proof}
    Let $s = \operatorname{sup}A$ and $t = \operatorname{sup}B$. We first show that $s + t$ is an upper bound for $A + B$. $a \leq s, \forall\ a \in A$, and $b \leq t, \forall\ b \in B$. This implies that $a + b \leq s + t$. Let u be an arbitrary bound for $A+B$. $a + b \leq u\implies b \leq u-a$, and since t is the least upper bound, this implies $t \leq u -a$. Same logic can be applied to get $a \leq u - b \implies s \leq u- b$. $s + t \leq 2u - a - b$, where $2u - a -b$ is an upper bound for A + B, therefore $s + t$ is the least upper bound.
\end{proof}

\end{document}