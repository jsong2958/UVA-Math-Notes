\documentclass{article}

\usepackage{amsmath}
\usepackage{amssymb}
\usepackage{enumitem}
\usepackage[margin=1in]{geometry}

% Commands
\newcommand{\term}[1]{\textbf{\emph{#1}}}
\newcommand{\mm}[1]{\(#1\)}

% Theorems
\newtheorem{definition}{Definition}[section]

% List

\newlist{axioms}{enumerate}{1}
\setlist[axioms]{label=(\arabic*), leftmargin=4em}


\title{Induction}
\author{Joseph Song}
\date{}

\begin{document}

\maketitle

\section{Induction}
Let $S \subseteq \mathbb{N}$. If S contains 1, and for any element $n \in S$, $n + 1 \in S$, then $S = \mathbb{N}$.
\\
\\
\textbf{Proof by induction:} show the base case ($n = 0, 1$). Then with the inductive hypothesis, the case where it holds for $n$, show it holds for $n + 1$. 

\begin{example}
Let $y_1 = 6$, and define $y_{n+1} = (2y_n-6)/3$.
\\
Prove $y_n > -6, \forall n \in \mathbb{N}$.
\end{example}
\begin{proof}
    First, we prove the base case, $(n=1)$. In this case, it is given. $y_1=6>-6$.
    Then, with the inductive hypothesis, show it holds for $n+1$.
    The inductive hypothesis is we claim $y_n > -6$, because it holds for $n$. Using that, derive $y_{n+1} > -6$ as well to complete the proof.
    \\
    \\
    With some scratch work: 
    \begin{align*}
         &y_{n+1} = (2y_n-6)/3 > -6 \\
         &\Rightarrow\quad 2y_n-6 > -18 \\
         &\Rightarrow\quad 2y_n > - 12 \\ 
         &\Rightarrow\quad y_n > -6
    \end{align*}
    We get $y_n > -6$ at the end, but we want to use that in the beginning. So our inductive step goes like this:
    \begin{align*}
       &y_n > - 6\\
       &\Rightarrow\quad 2y_n > -12 \\
       &\Rightarrow\quad 2y_n - 6 > -18 \\
       &\Rightarrow\quad (2y_n - 6)/3 > -6 \\
       &\Rightarrow\quad (2y_n - 6)/3 = y_{n+1} \\
       &\Rightarrow\quad y_{n+1} > -6
    \end{align*}
\end{proof}


\end{document}


