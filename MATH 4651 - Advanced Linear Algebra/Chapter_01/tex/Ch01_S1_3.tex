\documentclass{article}
 
 \usepackage{amsmath}
\usepackage{amssymb}
\usepackage{enumitem}
\usepackage[margin=1in]{geometry}

% Commands
\newcommand{\term}[1]{\textbf{\emph{#1}}}
\newcommand{\mm}[1]{\(#1\)}

% Theorems
\newtheorem{definition}{Definition}[section]

% List

\newlist{axioms}{enumerate}{1}
\setlist[axioms]{label=(\arabic*), leftmargin=4em}

 
 \title{Chapter 1, Section 1.3}
 \author{Joseph Song}
 \date{}
 
 \begin{document}
 
 \maketitle
 
 \section{Subspaces}
 
 \begin{definition}[Subspace]
 A subset $W$ of a vector space $V$ over a field $F$ is a \term{subspace} if $W$ is a vector space over $F$ with the operations defined on $V$. The subspace $\{ \emph{0} \}$ is called the \term{zero subspace} of $V$.
 \end{definition}
 
 The following is the theorem to prove if a subset is a subspace of $V$:
 \begin{theorem}
     Let $W$ be a subset of the vector space $V$. Then $W$ is a subspace of $V$ if and only if the following hold:
 \end{theorem}
 \begin{enumerate}[label=(\arabic*)]
     \item $0 \in W$
     \item $x + y \in W$, for $x, y \in W$
     \item $cx \in W$, for $c \in \mathbb{F}$ and $x \in W$
 \end{enumerate}
 \begin{example}[Polynomials]
 \end{example}
 For $n \in \mathbb{N}$, let $P_n(F)$ be the set containing all the polynomials in $P(F)$ having degree less than or equal to n.
 \\
 To show that it is a subspace, we use the theorem and see if the three condition holds. The zero polynomial has degree $-1 \leq n, \forall n \in \mathbb{N}$.
 The sum of two polynomials with degree less than or equal to n still has the degree less than or equal to n, along with scalar multiplication.
 
 \begin{example}[Functions]
 \end{example}
 
 Let $C(R)$ denote the set of all continuous real-valued functions defined on $\mathbb{R}$. Claim $C(R)$ is a subspace of $F(R, R)$.
 \\
 The zero function of $F(R,R)$ is just $f(t) = 0, \forall t\in \mathbb{R}$. But since constant functions are continuous, it is also an element of $C(R)$. Also notice that the sum of two continuous functions is continuous, along with the scalar multiplication of a continuous function, meaning they also belong in $C(R)$. So $C(R)$ is closed under addition and scalar multiplication, hence is a subspace of $F(R,R)$.
 
 \end{document}