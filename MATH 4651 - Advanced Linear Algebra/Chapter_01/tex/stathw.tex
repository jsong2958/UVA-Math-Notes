\documentclass{article}
 
\usepackage{amsmath}
\usepackage{amssymb}
\usepackage{enumitem}
\usepackage{amsfonts}
\usepackage{etoolbox}
\usepackage[margin=1in]{geometry}

\setlength{\parindent}{0pt}

% Commands
\newcommand{\term}[1]{\textbf{\emph{#1}}}
\newcommand{\mm}[1]{\(#1\)}

% Theorems
\newtheorem{defn}{Definition}[section]
\newenvironment{definition}[1]
{\begin{defn}{\term{{(#1).}}}}
{\end{defn}}

% List

\newlist{axioms}{enumerate}{1}
\setlist[axioms]{label=(\arabic*), leftmargin=4em}

 
\title{Chapter 1, Section 1.4}
\author{Joseph Song}
\date{}
 
\begin{document}

2. y = 1.6x - 74.9, r = 0.932
\\
\\
3. The slope is the successful screams per height in inches (screams / in). Meaning depending on the height of the monster, on average, the number of successful screams will change by the slope amount.
\\The y-intercept means if the monster's height is 0 inches, then they will have a negative amount of successful screams, meaning to have a successful scream, they must have some height.
\\
\\
4. The residual plot looks like they have no clear pattern, indicating that linear regression is a good choice for this data.
\\
\\
9. The linear regression line shows a correlation between height and the number of screams, so to improve the scare program, the dean should enroll taller monsters to the program, and that her claim was wrong.
\\
\\ 
10. Based on the equation of the line of regression, in one year, a 65 inch monster screams 29 times. So it would take $150/29 = 5.2$ years to graduate. For 75 inch monster, it would take $150/46 = 3.3$ years.
\\
\\
11. For a monster to graduate in 4 years or less, they need at least $150/4 = 37.5$ screams per year. So based on the linear of regression, x has to be at least 69.7 inches for a monster to have 37.5 screams per year. So only accept students 69.7 inches or taller.
\\
\\
12. $r^2 = 0.93$ and it means that there is a strong positive correlation between height and screams.

\end{document}