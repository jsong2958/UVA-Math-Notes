\documentclass{article}
 
\usepackage{amsmath}
\usepackage{amssymb}
\usepackage{enumitem}
\usepackage[margin=1in]{geometry}

% Commands
\newcommand{\term}[1]{\textbf{\emph{#1}}}
\newcommand{\mm}[1]{\(#1\)}

% Theorems
\newtheorem{definition}{Definition}[section]

% List

\newlist{axioms}{enumerate}{1}
\setlist[axioms]{label=(\arabic*), leftmargin=4em}

 
\title{Chapter 1, Section 1.4}
\author{Joseph Song}
\date{}
 
\begin{document}

 
\maketitle

\section{Linear Combinations}
\begin{definition}[Linear Combination]
    Let $S$ be a nonempty subset of the vector space $V$. A vector $v \in V$ is a \term{linear combination} of vectors in $S$ if there exists a finite numbers of vectors $u_1, u_2, \dots, u_n \in S$ and coefficient scalars $c_1, c_2,\dots, c_n \in \mathbb{F}$ such that $v = c_1v_1 + c_2v_2+\dots+c_nv_n$.
\end{definition}

Notice how the zero vector exists in any $V$, so the zero vector is a linear combination of the vectors in any $S$.

\begin{example}[Polynomials]
\end{example}

Is $2x^3 -2x^2+12x-6$ a linear combination of $x^3-2x^2-5x-3$ and $3x^3-5x^2-4x-9$?
\\
There must exists scalars $a$ and $b$ such that:
\[
2x^3 -2x^2+12x-6 = a(x^3-2x^2-5x-3) + b(3x^3-5x^2-4x-9)
\]
Distributing:
\[
2x^3 -2x^2+12x-6 = ax^3-2ax^2-5ax-3a + (3bx^3-5bx^2-4bx-9b)
\]
Setting the coefficients in a system of equations:

\begin{align*}
   a+3b=2 \\
   -2a-5b=2 \\
   -5a-4b=12 \\
   -3a-9b=-6
\end{align*}

Solve for $a$ and $b$:
%\p
\begin{align*}
\begin{array}{c}
    2a+6b=4 \\
   -2a-5b=2 \\
   \hline
   b = 2\\ 
   \\
   a + 3(2) = 2 \implies a = -4
\end{array}
\end{align*}


Plug into the other equations:
\[
-2(-4) - 5 (2) = 8 - 10 = -2 \neq 2
\]

We have reached a contradiction, so it is NOT a linear combination

\begin{example}[Vectors]
\end{example}

Is (2,1,9) a linear combination of (1,2,0) and (0,-1,3)?
\begin{align*}
    a = 2\\
    2a - b = 1\\
    3b = 9
\end{align*}

$a = 2$ and $b = 3$, so we verify:
\[
2(2) - 1(3) = 1
\]

So it is a linear combination.

\section{Span}
\begin{definition}
    Let $S$ be a nonempty subset of the vector space $V$. The \term{span} of $S$, denoted $span(S)$, is the set of all linear combinations of the vectors in $S$. Define $span(\emptyset) = 0$.
\end{definition}

\begin{theorem}
    The span of any subset of $V, S$, is a subspace of $V$. Additionally, I COOKED. \\

    please psssss
\end{theorem}

\end{document}