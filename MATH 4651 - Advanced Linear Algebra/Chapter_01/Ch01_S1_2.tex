\documentclass{article}

\usepackage{amsmath}
\usepackage{amssymb}
\usepackage{enumitem}
\usepackage[margin=1in]{geometry}

% Commands
\newcommand{\term}[1]{\textbf{\emph{#1}}}
\newcommand{\mm}[1]{\(#1\)}

% Theorems
\newtheorem{definition}{Definition}[section]

% List

\newlist{axioms}{enumerate}{1}
\setlist[axioms]{label=(\arabic*), leftmargin=4em}


\title{Chapter 1, Section 1.2}
\author{Joseph Song}
\date{}

\begin{document}

\maketitle

\section{Vector Spaces}

A vector taught in previous courses may have defined one as a quantity with a magnitude and direction, commonly represented as an arrow. Here, we define what a vector is rigorously.

\begin{definition}{Vector Space}
A \term{vector space} V over a field F is a set in which two operations, addition and scalar multiplication, are defined so that for each pair of elements \mm{x, y \in V}, the sum \mm{x + y} is also in V, and for any scalar \mm{c \in F}, the product cx is also an element of V, such that the following axioms hold:
\end{definition}

\begin{axioms}
    \item For all \mm{x, y \in V, x + y = y + x}
    \item For all \mm{x, y, z \in V, (x + y) + z = x + (y + z)}
    \item There exists the zero vector in V, denoted \mm{0}, such that \mm{x + 0 = x, \forall x \in V}
    \item For each \mm{x \in V, \exists y \in V} such that \mm{x + y = 0}
    \item For each x in V, \mm{1x = x}
    \item For each scalar \mm{a, b \in F} and each element \mm{x \in V}, \mm{(ab)x = a(bx)}
    \item For each element \mm{a \in F}, and each pair of elements \mm{x, y \in V}, \mm{a(x + y) = ax + ay}
    \item For each pair of scalars \mm{a, b \in F} and each element \mm{x \in V, (a + b)x = ax + bx}
\end{axioms}


\mm{\star} Elements of F are scalars, elements of V are vectors. Most cases, the vector space is over the field \mm{\mathbb{R}} or \mm{\mathbb{C}}\newline

\textbf{Vector Space Examples}
\begin{itemize}
    \item \textbf{N-tuples}: \mm{(a_1, a_2, a_3,...,a_n), a_n \in F}. The set of n-tuples with n entries of \mm{F} is the set \mm{F^n}
    \item \textbf{Matrices}: An m x n matrix with entry \mm{a_ij}, where m is the row and n is the column.
\end{itemize}




\end{document}